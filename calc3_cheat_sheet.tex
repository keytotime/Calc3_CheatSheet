\documentclass[10pt,landscape]{article}
\usepackage{multicol}
\usepackage{calc}
\usepackage{ifthen}
\usepackage[landscape]{geometry}
\usepackage{amsmath,amsthm,amsfonts,amssymb}
\usepackage{color,graphicx,overpic}
\usepackage{hyperref}
\usepackage{graphicx}
\usepackage{amsmath}
\usepackage{verbatim}

%Originally Modified by Daniel Kenner for Calc 3 Cheat Sheet
%Template Originally found @ http://tex.stackexchange.com/questions/8827/preparing-cheat-sheets
%Conic images originally found on google images when looking for named shapes
%Then modify Modified by Luca Buratto

%%%%%%%%%%%%%%%%%%%%%%%%%%%%%%%%%%5
%           ATTENZIONE
%%%%%%%%%%%%%%%%%%%%%%%%%%%%%%%%%%
% 19 righe disponobili al 10 maggio, senza eliminare altro e avendo documentato delle parti:
%10 righe nella penultima colonna foglio2
% 7 nell'ultima del foglio 2

\graphicspath{ {sheet_images/} }
%Originally Modified by Daniel Kenner for Calc 3 Cheat Sheet
%Template Originally found @ http://tex.stackexchange.com/questions/8827/preparing-cheat-sheets
%Conic images originally found on google images when looking for named shapes

\pdfinfo{
  /Title (Calculus 3 Cheat Sheet.pdf)
  /Creator (TeX)
  /Producer (pdfTeX 1.40.0)
  /Author (Daniel Kenner)
  /Subject (Example)
  /Keywords (pdflatex, latex,pdftex,tex)}

% This sets page margins to .5 inch if using letter paper, and to 1cm
% if using A4 paper. (This probably isn't strictly necessary.)
% If using another size paper, use default 1cm margins.
\ifthenelse{\lengthtest { \paperwidth = 11in}}
    { \geometry{top=.25in,left=.25in,right=.25in,bottom=.25in} }
    {\ifthenelse{ \lengthtest{ \paperwidth = 297mm}}
        {\geometry{top=1cm,left=1cm,right=1cm,bottom=1cm} }
        {\geometry{top=1cm,left=1cm,right=1cm,bottom=1cm} }
    }

% Turn off header and footer
\pagestyle{empty}

% Redefine section commands to use less space
\makeatletter
\renewcommand{\section}{\@startsection{section}{1}{0mm}%
                                {-1ex plus -.5ex minus -.2ex}%
                                {0.5ex plus .2ex}%x
                                {\normalfont\large\bfseries}}
\renewcommand{\subsection}{\@startsection{subsection}{2}{0mm}%
                                {-1explus -.5ex minus -.2ex}%
                                {0.5ex plus .2ex}%
                                {\normalfont\normalsize\bfseries}}
\renewcommand{\subsubsection}{\@startsection{subsubsection}{3}{0mm}%
                                {-1ex plus -.5ex minus -.2ex}%
                                {1ex plus .2ex}%
                                {\normalfont\small\bfseries}}
\makeatother

% Define BibTeX command
\def\BibTeX{{\rm B\kern-.05em{\sc i\kern-.025em b}\kern-.08em
    T\kern-.1667em\lower.7ex\hbox{E}\kern-.125emX}}

% Don't print section numbers
\setcounter{secnumdepth}{0}


\setlength{\parindent}{0pt}
\setlength{\parskip}{0pt plus 0.5ex}

%My Environments
\newtheorem{example}[section]{Example}
% -----------------------------------------------------------------------

\begin{document}
\raggedright
\footnotesize
\begin{multicols*}{5}

% multicol parameters
% These lengths are set only within the two main columns
%\setlength{\columnseprule}{0.25pt}
\setlength{\premulticols}{1pt}
\setlength{\postmulticols}{1pt}
\setlength{\multicolsep}{1pt}
\setlength{\columnsep}{2pt}

\section{Regole derivate}
\scriptsize
$\mathrm{D}[\alpha f(x)+ \beta g(x)] = \alpha f'(x) + \beta g'(x) \qquad$\newline
$\mathrm{D} [ {f(x) \cdot g(x)}] = f'(x) \cdot g(x) + f(x) \cdot g'(x)$ \newline
$ \mathrm{D} \left[ {\frac{f(x)}{g(x)}} \right] = { f'(x)  \cdot g(x) - f(x) \cdot   \frac{g'(x)}{g(x)^2}}$\newline
$\mathrm{D} \left[ {1 \over f(x)} \right] = - \frac{f'(x)} {f(x)^2} $\newline
$ \mathrm{D}[f^{-1}(y)]  =  {1 \over f'(x)}$\newline
$ \mathrm{D} \left[ f \left( g(x) \right) \right] = f' \left( g(x) \right) \cdot g'(x)$\newline
\section{Derivate fondamentali}
$D_x e^x=e^x$\newline
$D_x \sin(x)=\cos(x)$\newline
$D_x \cos(x)=-\sin(x)$\newline
$D_x \tan(x)=\frac{\sin(x)}{\cos(x)}$\newline
$D_x \cot(x)=-\csc^2(x)$\newline

$D_x \sin^{-1}=\frac{1}{\sqrt{1-x^2}}, x \in [-1,1]$\newline
$D_x \cos^{-1}=\frac{-1}{\sqrt{1-x^2}}, x \in [-1,1]$\newline
$D_x \tan^{-1}=\frac{1}{1+x^2}, \frac{-\pi}{2}\le x \le \frac{\pi}{2}$\newline
$D_x \sec^{-1}=\frac{1}{\mid x \mid \sqrt{x^2-1}}, |x| > 1$\newline


$ D_x \ln(x) = \frac{1}{x} $

\section{Integrali}
\scriptsize
$\int \frac{1}{x}dx = \ln|x|+c$\newline
$\int e^x dx = e^x+c $\newline
$\int a^x dx = \frac{1}{\ln a} a^x+c $\newline
$\int e^{ax} dx = \frac{1}{a} e^{ax}+c $\newline
$\int \frac{1}{\sqrt{1-x^2}} dx = \sin^{-1}(x)+c $\newline
$\int \frac{1}{1+x^2} dx = \tan^{-1}(x)+c $\newline


$\int \tan(x) dx = -\ln|\cos(x) |+c $\newline
$\int \cot(x) dx = \ln|\sin(x)|+c $\newline
$\int \cos(x) dx = \sin(x)+c $\newline
$\int \sin(x) dx = -\cos(x)+c $\newline
$\int \frac{1}{\sqrt{a^2-u^2}} dx = \sin^{-1}(\frac{u}{a})+c $\newline
$\int \frac{1}{a^2+u^2} dx = \frac{1}{a}\tan^{-1}\frac{u}{a}+c $\newline
$ \int ln(x) dx = (xln(x))-x+c $\newline

\textbf{Integrazione per sostituzione}\newline
Sia $ u=f(x) $ (pu\`o essere pi\`u di una variabile).\newline
Determina: $ du = \frac{f(x)}{dx}dx $ e risoli per  dx.\newline
Poi, se l'integrale \`e definito, sostituisci i confini per $ u=f(x) $ per ciascun confine\newline

Risolvi l'integrale usando u.\newline

\textbf{Integrazione per parti}\newline
$ \int u dv = uv-\int v du $

\section{Sezione}

\section{Identit\`a trigonometriche}
$ \sin^2(x)+\cos^2(x) = 1 $\newline

$ \sin(x\pm y) = \sin(x)\cos(y)\pm\cos(x)\sin(y) $\newline
$ \cos(x\pm y) = \cos(x)\cos(y)\pm\sin(x)\sin(y) $\newline
$ \tan(x\pm y) = \frac{\tan(x)\pm\tan(y)}{1 \mp \tan(x)\tan(y)} $\newline
$ \sin(2x) = 2\sin(x)\cos(x) $\newline
$ \cos(2x) = \cos^{2}(x) - \sin^{2}(x) $\newline

$ \sin^2(x) = \frac{1-\cos(2x)}{2} $\newline
$ \cos^2(x) = \frac{1+\cos(2x)}{2} $\newline
$ \tan^2(x) = \frac{1-\cos(2x)}{1+\cos(2x)} $\newline
$ \sin(-x) = -\sin(x) $\newline
$ \cos(-x) = \cos(x) $\newline
$ \tan(-x) = -\tan(x) $

\section {Calculo 3 Concetti}

\subsection{Coordinate cartesiane 3D}
dati due punti:\newline
$ (x_1, y_1, z_1) $ and $ (x_2, y_2, z_2)$,\newline
Distanza fra di loro :\newline
$ \sqrt{(x_1-x_2)^2+(y_1-y_2)^2+(z_1-z_2)^2} $\newline
Pt medio:\newline
$ (\frac{x_1+x_2}{2},\frac{y_1+y_2}{2},\frac{z_1+z_2}{2}) $\newline
Sfera di centro (h,k,m) and radius r:\newline
$ (x-h)^2 + (y-k)^2 + (z-m)^2 = r^2 $

\subsection{Vettori}
Vettore: $ \vec{u} $\newline
Vettore unitario: $ \hat{u} = \frac{\vec{u}}{||\vec{u}||}$\newline
Norma: $ ||\vec{u}|| = \sqrt{u_1^2+u_2^2+u_3^2}$\newline


\textbf{Prodotto scalare}\newline
$ \vec{u} \cdot \vec{v} $\newline
produce uno Scalare  \newline
(geometricamente, il prodotto scalare \`e il vettore proiezione)\newline
$\vec{u} = < u_1, u_2, u_3 >$ \newline
$\vec{v} = < v_1, v_2, v_3 >$\newline
se $ \vec{u} \cdot \vec{v} = \vec{0} $ significa che i due vettori sono Perpendicolari, 
$ \theta $ \`e l'angolo compreso fra loro.\newline
$\vec{u} \cdot \vec{v} = ||\vec{u}||\:||\vec{v}||\cos(\theta) $\newline
$\vec{u} \cdot \vec{v} = u_1v_1 + u_2v_2 + u_3v_3 $\newline
NOTE:\newline
$ \hat{u} \cdot \hat{v} = \cos(\theta) $\newline
$ ||\vec{u}||^2 = \vec{u} \cdot \vec{u} $\newline
$ \vec{u} \cdot \vec{v} = 0 $ quando $ \bot $\newline
Angolo fra $ \vec{u} $ e $ \vec{v} $:\newline
$ \theta = \cos^{-1}(\frac{\vec{u} \cdot \vec{v}}{||\vec{u}||\:||\vec{v}||}) $\newline
Proiezione di  $ \vec{u} $ su  $ \vec{v} $:\newline
$ pr_{\vec{v}}\vec{u} = (\frac{\vec{u} \cdot \vec{v}}{||\vec{v}||^2})\vec{v} $\newline

\textbf{Prodotto vettoriale}\newline
$\vec{u} \times \vec{v}$\newline
Produce un Vettore\newline
(Geometricamente, il prodotto vettoriale = l'area del paralellogramma di dimensioni $ ||\vec{u}|| $ e  $ ||\vec{v}|| $)\newline
$\vec{u} = < u_1, u_2, u_3 >$\newline
$\vec{v} = < v_1, v_2, v_3 >$\newline
\[
\vec{u} \times \vec{v} = 
\begin{vmatrix}
\hat{i} & \hat{j} & \hat{k} \\
u_1 & u_2 & u_3 \\
v_1 & v_2 & v_3
\end{vmatrix}
\]\newline 
$ \vec{u} \times \vec{v} = \vec{0} $ significa che i vettori sono paralleli 

\subsection {Linee e Piani}
\textbf{Equazione del Plino}\newline
$ (x_0, y_0, z_0) $ \`e un punto sul piano e $ <A,B,C> $ \`e un vettore normale \newline

$A(x-x_0)+B(y-y_0)+C(z-z_0) = 0$\newline
$ <A,B,C> \cdot <x-x_0, y-y_0, z-z_0> = 0 $\newline
$ Ax+By+Cz = D $ dove $ D=Ax_0+By_0+Cz_0 $\newline

\textbf{Equazione di una linea}\newline
Una linea richiede un Vettore Direzionale $ \vec{u}=<u_1,u_2,u_3> $ e un punto $(x_1,y_1,z_1)$\newline
poi,\newline le parametrizzazioni di una linea  possono essere:\newline
$ x = u_1t+x_1 $\newline
$ y = u_2t+y_1 $\newline
$ z = u_3t+z_1 $\newline

\textbf{Distanza di un Punto da un Piano}\newline
la distanza di un punto $(x_0,y_0,z_0)$ da un piano  Ax+By+Cz=D la posso esprimere attraverso la formula:\newline
$ d=\frac{|Ax_0+By_0+Cz_0-D|}{\sqrt{A^2+B^2+C^2}} $\newline


\subsection {Legami fra teoremi }
\textbf{Differenziabilit\`a,  derivabilit\`a e continuit\`a}\newline
TDT: se esiste intorno di un pt in cui $f(x,y)$ ha derivate parziali continue nel pt, allora $f(x,y)$  differenziabile nel pt. \newline
- Se $f(x,y)$ differenziabile nel pt, allora $f(x,y)$ è continua nel pt. (no viceversa) \newline
- Se $f(x,y)$ differenziabile nel pt, allora $f(x,y)$ parzialmente derivabile nel pt. (no viceversa) \newline
- no legami fra derivate parziali e contnuit\`a

\textbf{Teor Schwarz}\newline
Se $f(x,y)$ definita in almeno un intorno del pt e sono definite e continue nel pt le derivate parziali prime, allora le derivate parziali seconde $f_{xy}=f_{yx}$ \newline
\textbf{Titoletto}\newline

\subsection{Superfici}
\textbf{Ellissoide}\newline
$ \frac{x^2}{a^2}+\frac{y^2}{b^2}+\frac{z^2}{c^2} = 1 $\newline
\includegraphics[scale=0.15]{ellipsoid}\newline
\textbf{Iperboloide di Un Foglio }\newline
$ \frac{x^2}{a^2}+\frac{y^2}{b^2}-\frac{z^2}{c^2} = 1 $\newline
(Asse maggiore: asse Z in quanto non ha -)\newline
\includegraphics[scale=0.15]{hyperboloid1}\newline
\textbf{Iperboloide di Due Fogli }\newline
$ \frac{z^2}{c^2}-\frac{x^2}{a^2}-\frac{y^2}{b^2} = 1 $\newline
(Asse maggiore: asse Z in quanto unico che non ha -)\newline
\includegraphics[scale=0.15]{hyperboloid2}\newline
\textbf{Paraboloide Ellittico}\newline
$ z=\frac{x^2}{a^2}+\frac{y^2}{b^2} $\newline
(Asse maggiore: Z in quanto variabile NON elevato
\includegraphics[scale=0.15]{elliptic_paraboloid}\newline
\textbf{Paraboloide Iperbolico}\newline
(Asse maggiore: asse Z in quanto non elevato al quadrato)\newline
$ z=\frac{y^2}{b^2}-\frac{x^2}{a^2} $\newline
\includegraphics[scale=0.15]{hyperbolic_paraboloid}\newline
\textbf{Cono Ellittico}\newline
(Asse Maggiore: asse Z in quanto unico che ha segno -)\newline
$ \frac{x^2}{a^2}+\frac{y^2}{b^2}-\frac{z^2}{c^2} = 0 $\newline
\includegraphics[scale=0.15]{elliptic_cone}\newline
\textbf{Cilindro}\newline
se manca una delle variabili \newline
OPPURE\newline
$ (x-a)^2+(y-b^2) = c $\newline
(Asse Maggiore \`e la variabile mancante)

\subsection{Derivate Parziali}
Si calcolano semplicemente tenendo ferme tutte le altre variabili (che si comportano come costanti per la derivata) e si calcola solo la derivata rispetto a una determinata variabile.\newline
Data z=f(x,y), la derivata parziale di z rispetto alla variabile x \`e:\newline
$f_x(x,y) = z_x = \frac{\partial z}{\partial x} = \frac{\partial f(x,y)}{\partial x}$\newline
idem per la derivata parziale rispetto ad y:\newline
$f_y(x,y) = z_y = \frac{\partial z}{\partial y} = \frac{\partial f(x,y)}{\partial y}$\newline
\textbf{Notazione}\newline
Per $ f_{xyy} $, opera da dentro verso fuori  $ f_{x} $ then $ f_{xy} $, then $ f_{xyy} $\newline
$f_{xyy} = \frac{\partial^3 f}{\partial x \partial^2 y}, $\newline
Per $\frac{\partial^3 f}{\partial x \partial^2 y}$ , opera da destra a sinistra nel denominatore

\subsection {Gradiente}
Il Gradiente di una funzione in 2 variabili si indica con $ \nabla f = < f_x, f_y > $\newline
Il Gradiente di una funzione in 3 variabili si indica con $ \nabla f = < f_x, f_y, f_z > $

\subsection {Regola(e) della Catena}
no

\subsection{Limiti e Continuit\`a}
\textbf{Limiti in 2 o pi\`u variabili}\newline

I limiti rilevati su un limite vettoriale possono essere valutati separatamente per ciascun componente del limite.\newline
\textbf{Strategie per mostrare che il limite esiste}\newline
1. Inserire i numeri, tutto a posto\newline
2. Manipolazioni Algebriche\newline
-fattorizzare\ dividere \newline
-usa identità triangolari \newline
3. Cambia in coord polari \newline
se \:\:$ (x,y)\to(0,0)\Leftrightarrow r\to0$\newline
\textbf{Strategie per mostrare che limite NE}\newline
1. Mostra che il limite \`e  diverso se approssimato da percorsi diversi \newline
(x=y, $x=y^2,$ etc.)\newline
2. Cambia in coord Polari e mostra che il limite NE.\newline
\textbf{Continuit\`a}\newline
Una fn, $z=f(x,y)$, \`e  continua in (a,b) se \newline
$f(a,b) = \lim_{(x,y) \to (a,b)} f(x,y) $\newline
Che significa:\newline
1. Il limite esiste\newline
2. La funzione \`e definita in quel valore\newline
3. Essi hanno lo stesso valore

\subsection{ Derivate Direzionali}
Sia z=f(x,y) una funzione, (a,b) un punto nel  dominio (un valido  punto di input) e  $ \hat{u} $ un vettore unitario (2D).\newline
La Derivata Direzionale \`e quindi la la derivata nel punto (a,b) con direzione di $ \hat{u} $ o:\newline
$ D_{\vec{u}}f(a,b) = \hat{u} \cdot \nabla f(a,b)$\newline
Ciò restituir\`a uno \textit{scalare}. \newline
%4-D version: $ D_{\vec{u}}f(a,b,c) = \hat{u} \cdot \nabla f(a,b,c) $

\subsection{Piani Tangenti}
Sia F(x,y,z) = k una superfice e  P = $(x_0, y_0, z_0)$ un punto su questa superficie.\newline
L'equazione del Piano Tangente \`e \newline
$ \nabla F(x_0,y_0,z_0) \cdot <x-x_0, y-y_0, z-z_0> $

\subsection {Approssimazioni}
Sia $ z=f(x,y)$ una funzione differenziabile differenziale totale di f = dz\newline
$dz= \nabla f \cdot < dx, dy >$\newline
Questa \`e la variazione \textit{approssimata} in z\newline
Il cambiamento reale  in z \`e  la differenza nei valori di z:\newline
$ \Delta z = z-z_1 $

\newpage

\subsection {Massimi e Minimi}
$\bullet$ Trovare estremi relativi di funzioni definite su insiemi \underline{aperti}, derivabili ovunque \newline
$\bullet$ Trovare estremi \underline{assoluti} di funzioni continue e derivabili ovunque, definite su insiemi chiusi e limitati. Si risolve in 3 passi:
\begin{itemize}
    \item[$a)$] cerco pt di max/min interni (via annullamento delle derivate prime)
    \item[$b)$] cerco pt di max/min sulla frontiera (via sostituzione o ML)
    \item[$c)$] calcolo il valore della funzione in tutti i pt trovati in a) e b). Fra questi pt ci sono certamente i pt di max e min assoluto, che $\exists$ per teor. Weierstrass 
\end{itemize}
$\bullet$ Trovare estremi relativi di funzione con vincolo
\begin{itemize}
\item[-] si risolve via sostituzione. I ML non bastano, danno condizioni solo necessarie
\end{itemize}

\textbf{Risultati per punti interni}\newline
- c. necess: derivate parziali nel pt siano nulle \newline
\textbf{Utile su D aperto}\newline
- c. suff: se $f(x,y)$ ammette derivate parziali prime e seconde continue in un intorno del pt e vale la c. necess., allora: \newline
a) c. suff. per cui pt sia pt minimo relativo interno \`e che
$\left\{\begin{matrix}
f_{xx}(pt)>0\\ 
|H(pt)|>0
\end{matrix}\right.$
\newline
a') c. suff. per cui pt sia pt massimo relativo interno \`e che 
$\left\{\begin{matrix}
f_{xx}(pt)<0\\ 
|H(pt)|>0
\end{matrix}\right.$
- non concludo nulla se $|H(pt)|\geq0$
\newline
- un pt NON \`e pt di ottimo se  $|H(pt)|<0$ \newline
\newline
\textbf{Max e min su frontiera del dominio} \newline
min/max su $D \cap \{(x,y): g(x,y)=0\}$ \newline
I)metodo di sostituzione: se riesco esplicito $g(x,y)=0$ risp x o y ottenendo problema di max/min in una var soltanto
\newline
II)metodo dei moltiplicatori di Lagrange: fornisce cond. necess. per punti di ottimo sul vincolo $g(x,y)$ vedi Moltiplicatori di Lagrange
\newline
\newline
\textbf{Punti Interni}\newline
1. Calcola le Derivate Parziali rispetto a x e y ($ f_x $ e $ f_y $) (Puoi usare il gradiente)\newline
2. Poni le derivate uguali a 0 e risolvi il sistema di equazioni rispetto x e y\newline
3. Inserisci nell'equazione originaria in z.\newline
Usa Test delle Derivate Seconde per sapere se i punti sono massimi o minimi locali, o sella\newline

\textbf{Test delle Derivate Seconde Parziali}\newline
1. Trova tutti i punti (x,y) per cui $ \nabla f(x,y) = \vec{0} $\newline
2. Sia $ D=f_{xx}(x,y)f_{yy}(x,y)-f_{xy}^2(x,y) $\newline
SE (a) D $>$ 0 ET $ f_{xx} < 0, $ f(x,y) \`e massimo locale \newline
(b) D $>$ 0 ET $f_{xx}(x,y) > 0$ f(x,y) \`e minimo locale \newline
(c) D $<$ 0, (x,y,f(x,y)) \`e un punto di sella \newline
(d) D $=$ 0, test  inconclusivo \newline
3. Determina se ciascun punto sul confine fornisce un min o max. Tipicamente, dobbiamo parametrizzare il confine per  ridurlo ad un problema di min/max di Analisi 1.

\textbf{Quanto segue si applica solo se viene fornito un confine}\newline
1. Controlla  i punti d'angolo \newline
2. Controlla  ciascuna linea  (0 $\le$ x $\le$ 5 darebbe x=0 e x=5 )\newline
Sulle Equazioni Limitate, questo \`e il massimo/ minimo globale ... il test delle derivate seconde non \`e necessario.

\subsection{Moltiplicatori di Lagrange}
Data una funzione f(x,y) con il vincolo g(x,y), risolvi il seguente sistema di equazioni per trovare i punti di massimo e di minimo sul vincolo (NOTA: potrebbe essere necessario trovare anche punti interni):\\

$\left\{\begin{matrix}
 \nabla f = \lambda \nabla g  \\ 
 g(x,y) = 0 &(\mathrm{o }\, k \, \mathrm{se}\, {\`e}\,  \mathrm{dato})  
\end{matrix}\right.$\newline
Aggiungi eventuali pt singolari di $g(x,y)$ per cui $g(x,y)=0$ \newline
\newline
\textbf{Teor risparmia tempo su ML} \newline
Se $f(x,y)=h(r(x,y))$, con $u=r(x,y)$ e se $h'(u)>0$, allora i pt di massimo della $f(x,y)$ sono gli stessi di quelli della $g(x,y)$. Lo stesso per i pt di minimo.

\subsection{Integrali Doppi}
Rispetto all'asse xy, se prendiamo un integrale,\newline
$ \int\int dy dx $ sta tagliando in rettangoli verticali,\newline
$ \int\int dx dy $ sta tagliando in rettangoli orizzontali \newline

\textbf{Coordinate Polari}\newline
Quando usiamo coordinate polari, $ dA = r dr  d\theta $

\subsection{Area di superficie di una curva}
Sia z = f(x,y) continua su S (a regione chiusa in dominio 2D)\newline
L'area di superficie della z = f(x,y) su S \`e:\newline
$ SA = \int\int_S\sqrt{f_x^2+f_y^2+1}\, dA $

\subsection {Integrali Tripli}
$ \int\int\int_s f(x,y,z)dv = \int_{a_1}^{a_2}\int_{\phi_1(x)}^{\phi_2(x)}\int_{\psi_1(x,y)}^{\psi_2(x,y)} f(x,y,z)dzdydx $\newline
Nota: $ dv $ pu\`o essere  scambiato in $ dxdydz $ in qualsiasi ordine, tuttavia bisogna scegliere i limiti di integrazione rispettando l'ordine scelto

\subsection {Metodo Jacobiano}
$ \int\int_Gf(g(u,v),h(u,v)) | J(u,v) | du dv = \int\int_R f(x,y) dx dy $\newline
\[
J(u,v) = 
\begin{vmatrix}
\frac{\partial x}{\partial u} & \frac{\partial x}{\partial v} \\
\frac{\partial y}{\partial u} & \frac{\partial y}{\partial v}
\end{vmatrix}
\]\newline
%Jacobiani Comuni:\newline
%Rect. to Cylindrical: $r$\newline
%Rect. to Spherical: $\rho^2 \sin(\phi)$

%\subsection {Vector Fields}
%note, this one should be moved around so that it is on the last column...it's kinda big.
%let $ f(x,y,z) $ be a scalar field and $ \vec{F}(x,y,z) = M(x,y,z)\hat{i} + N(x,y,z)\hat{j} + P(x,y,z)\hat{k} $ be a vector field,\newline
%Grandient of f = $ \nabla f = < \frac{\partial f}{\partial x}, \frac{\partial f}{\partial y}, \frac{\partial f}{\partial z} >$\newline
%Divergence of $ \vec{F}$:\newline $ \nabla\cdot\vec{F} = \frac{\partial M}{\partial x} + \frac{\partial N}{\partial y} + \frac{\partial P}{\partial z} $\newline
%Curl of $ \vec{F}$:\newline $ \nabla\times\vec{F} = 
%\begin{vmatrix}
%\hat{i} & \hat{j} & \hat{k} & \\
%\frac{\partial}{\partial x} & \frac{\partial}{\partial y} & %\frac{\partial}{\partial z} \\
%M & N & P
%\end{vmatrix} $

\subsection{Line Integrals}
C given by $x = x(t), y = y(t), t\in [a,b]$\newline 
$ \int_cf(x,y)ds = \int_a^bf(x(t), y(t)) ds $\newline
where $ ds = \sqrt{(\frac{dx}{dt})^2+(\frac{dy}{dt})^2} dt $\newline
or $ \sqrt{1+(\frac{dy}{dx})^2} dx $\newline
or $ \sqrt{1+(\frac{dx}{dy})^2} dy $\newline
To evaluate a Line Integral,\newline
$\cdot$ get a paramaterized version of the line (usually in terms of t, though in exclusive terms of x or y is ok)\newline
$\cdot$ evaluate for the derivatives needed (usually dy, dx, and/or dt)\newline
$\cdot$ plug in to original equation to get in terms of the independant variable\newline
$\cdot$ solve integral\newline

\textbf{Work}\newline
Let $ \vec{F} = M\hat{i}+\hat{j}+\hat{k} $ (force) $ M=M(x,y,z), N=N(x,y,z), P=P(x,y,z) $\newline
(Literally)$ d\vec{r} = dx\hat{i}+dy\hat{j}+dz\hat{k} $\newline
Work $ w = \int_c\vec{F}\cdot d\vec{r} $\newline
(Work done by moving a particle over curve C with force $ \vec{F} $)

\subsection{Independence of Path}
\textbf{Fund Thm of Line Integrals}\newline
C is curve given by $ \vec{r}(t), t\in[a,b];\newline \vec{r}\prime(t) $ exists. If $ f(\vec{r}) $ is continuously differentiable on an open set containing C, then $ \int_c\nabla f(\vec{r}) \cdot d\vec{r} = f(\vec{b}) - f(\vec{a}) $\newline
\textbf{Equivalent Conditions}\newline
$ \vec{F}(\vec{r}) $ continuous on open connected set D. Then,\newline
$ (a) \vec{F} = \nabla f $ for some fn f. (if $ \vec{F} $ is conservative)
$ \Leftrightarrow (b) \int_c\vec{F}(\vec{r})\cdot d\vec{r} is indep. of path in D $\newline
$ \Leftrightarrow (c) \int_c\vec{F}(\vec{r})\cdot d\vec{r} = 0 $ for all closed paths in D.\newline
\textbf{Conservation Theorem}\newline
$ \vec{F} = M\hat{i}+N\hat{j}+P\hat{k} $ continuously differentiable on open, simply connected set D.\newline
$ \vec{F} $ conservative $ \Leftrightarrow \nabla\times\vec{F} = \vec{0} $ \newline
(in 2D $ \nabla\times\vec{F}=\vec{0} $ iff $ M_y = N_x$)

\subsection{Green's Theorem}
(method of changing line integral for double integral - Use for Flux and Circulation across 2D curve and line integrals over a closed boundary)\newline
$ \oint Mdy - Ndx = \int\int_R(M_x+N_y)dxdy $\newline
$ \oint Mdx + Ndy = \int\int_R (N_x-M_y) dxdy $\newline
Let:\newline
$\cdot$R be a region in xy-plane\newline
$\cdot$C is simple, closed curve enclosing R (w/ paramerization $ \vec{r}(t) $)\newline
$ \cdot\vec{F}(x,y) = M(x,y)\hat{i} + N(x,y)\hat{j} $ be continuously differentiable over R$\cup$C. \newline
\textbf{Form 1: Flux Across Boundary}\newline
$ \vec{n} = $ unit normal vector to C\newline
$ \oint_c \vec{F}\cdot\vec{n} = \int\int_R \nabla\cdot\vec{F} dA $\newline
$ \Leftrightarrow\oint Mdy - Ndx = \int\int_R(M_x+N_y)dxdy $\newline
\textbf{Form 2: Circulation Along Boundary}\newline
$ \oint_c\vec{F}\cdot d\vec{r} = \int\int_R \nabla\times\vec{F}\cdot\hat{u} dA $\newline
$ \Leftrightarrow \oint Mdx + Ndy = \int\int_R (N_x-M_y) dxdy $\newline
\textbf{Area of R}\newline
$ A = \oint(\frac{-1}{2}y dx + \frac{1}{2}x dy) $

\begin{comment}
 \subsection{Gauss' Divergence Thm}
(3D Analog of Green's Theorem - Use for Flux over a 3D surface)
Let:\newline
$ \cdot\vec{F}(x,y,z) $ be vector field continuously differentiable in solid S\newline
$ \cdot $S is a 3D solid
$ \cdot\partial S $ boundary of S (A Surface)\newline
$ \cdot\hat{n} $unit outer normal to $ \partial S $\newline
Then,\newline
$ \int\int_{\partial S}\vec{F}(x,y,z)\cdot\hat{n}dS = \int\int\int_S\nabla\cdot\vec{F} dV $\newline
(dV = dxdydz)
\end{comment}

% prova a caso
\subsection{prova a caso}
C given by $x = x(t), y = y(t), t\in [a,b]$\newline 
$ \int_cf(x,y)ds = \int_a^bf(x(t), y(t)) ds $\newline
where $ ds = \sqrt{(\frac{dx}{dt})^2+(\frac{dy}{dt})^2} dt $\newline
or $ \sqrt{1+(\frac{dy}{dx})^2} dx $\newline
or $ \sqrt{1+(\frac{dx}{dy})^2} dy $\newline
To evaluate a Line Integral,\newline
$\cdot$ get a paramaterized version of the line 
ultime parole ultime parole ultime parole ultime parole ultime parole ultime parole ultime parole ultime parole ultime parole ultime parole ultime parole ultime parole ultime parole 
parole ultime parole ultime parole ultime parole 
 ultime parole ultime parole ultime parole ultime parole ultime parole ultime parole ultime parole ultime parole ultime parole ultime parole ultime  ultime parole ultime parole ultime parole ultime parole 
%fine prova a caso1

\vfill
\subsection{Circonferenza unitaria}
(cos, sin)\newline
\includegraphics[scale=0.64]{unit_circle}
\newline\newline\newline\newline
\vfill
\columnbreak


\subsection{Surface Integrals}
Let\newline
$\cdot$R be closed, bounded region in xy-plane\newline
$\cdot$f be a fn with first order partial derivatives on R\newline
$\cdot$G be a surface over R given by $ z=f(x,y) $\newline
$\cdot g(x,y,z) = g(x,y,f(x,y))$ is cont. on R\newline
Then,\newline$ \int\int_G g(x,y,z) dS = \int\int_R g(x,y,f(x,y)) dS $\newline
where $ dS = \sqrt{f_x^2+f_y^2+1}dydx $\newline
\textbf{Flux of $ \vec{F} $ across G}\newline
$ \int\int_G\vec{F}\cdot{n} dS = \int\int_R[-Mf_x-Nf_y+P]dxdy $\newline
where:\newline
$\cdot\vec{F}(x,y,z) = M(x,y,z)\hat{i} + N(x,y,z)\hat{j} + P(x,y,z)\hat{k} $\newline
$\cdot$G is surface f(x,y)=z\newline
$\cdot\vec{n}$ is upward unit normal on G.\newline
$\cdot$f(x,y) has continuous $1^{st}$ order partial derivatives
 ultime parole ultime parole ultime parole ultime parole ultime parole ultime parole ultime parole ultime parole ultime parole ultime parole ultime  ultime parole ultime parole ultime parole ultime parole ultime parole ultime parole ultime parole ultime parole ultime parole ultime parole ultime 

\begin{comment}
 \subsection {Other Information}
$ \frac{\sqrt{a}}{\sqrt{b}} = \sqrt{\frac{a}{b}} $\newline
Where a Cone is defined as $ z = \sqrt{a(x^2+y^2)}, $\newline
In Spherical Coordinates, $ \phi = \cos^{-1}(\sqrt{\frac{a}{1+a}}) $\newline
Right Circular Cylinder:\newline
$ V=\pi r^2h, SA=\pi r^2+2\pi rh $\newline
$ \lim_{n\to \inf} (1+\frac{m}{n})^{pn} = e^{mp} $\newline
Law of Cosines:\newline
$ a^2 = b^2 + c^2 - 2bc(\cos(\theta)) $

\subsection{Stokes Theorem}
Let:\newline
$\cdot$S be a 3D surface\newline
$\cdot\vec{F}(x,y,z) = M(x,y,z)\hat{i}+N(x,y,z)\hat{j}+P(x,y,z)\hat{l} $\newline
$\cdot$M,N,P have continuous $1^{st}$ order partial derivatives\newline
$\cdot$C is piece-wise smooth, simple, closed, curve, positively oriented\newline
$\cdot\hat{T}$ is unit tangent vector to C.\newline
Then,\newline
$ \oint\vec{F}_c\cdot\hat{T}dS = \int\int_s (\nabla\times\vec{F})\cdot\hat{n} dS = \int\int_R(\nabla\times\vec{F})\cdot\vec{n}dxdy $\newline
Remember:\newline
$ \oint\vec{F}\cdot\vec{T}ds = \int_c (Mdx + Ndy + Pdz) $
\end{comment}

%The following can go away as needed, it is here so I can put my Accreditation message/source code message at the bottom right.
\vfill
{\fontsize{.5}{1}\selectfont Originally Written By Daniel Kenner for MATH 2210 at the University of Utah.\newline Source code available at https://github.com/keytotime/Calc3\_CheatSheet\newline Later modified and customized by Luca Buratto} 
\end{multicols*}
\end{document}
